% --- CLASE SOBERANA: IEEEtran ---
% 'journal' = Formato de revista (doble columna, denso).
% 'compsoc' = Optimizado para computación/ingeniería.
\documentclass[journal, compsoc, 12pt]{IEEEtran}

% --- ARSENAL DE IDIOMA Y FUENTE ---
\usepackage[utf8]{inputenc}
\usepackage[spanish, es-tabla]{babel} % 'es-tabla' usa "Tabla" en lugar de "Cuadro"
\usepackage{libertinus} % Tu fuente de autoridad
\usepackage[T1]{fontenc}
\usepackage{microtype}  % Micro-ajustes tipográficos esenciales para doble columna

% --- ARSENAL DE CITACIÓN(Vancouver/Numérico) ---
% IEEE usa citas numéricas [1], ideal para esto.
\usepackage[backend=biber, style=ieee, sorting=none]{biblatex}
\addbibresource{referencias.bib}

% --- UTILIDADES ---
\usepackage{graphicx}
\usepackage{float}
\usepackage{amsmath} % Para tus ecuaciones de señales
\usepackage{booktabs} % Para tablas profesionales
\usepackage[colorlinks=true, linkcolor=black, citecolor=black, urlcolor=blue]{hyperref}

% --- METADATOS DEL PAPER ---
\title{Análisis de Señales para la Detección de Patologías en Voz: Proyecto Cordectomía}

% En IEEE, los autores se ponen en un bloque especial
\author{Alfonso~Gamboa~Rubén% <- Este % evita espacios extra
\thanks{Manuscrito entregado el \today. Trabajo realizado para la materia de Análisis de Señales, Universidad Veracruzana.}}

% ==========================================================
%                  INICIO DEL PAPER
% ==========================================================
\begin{document}

% Crea el título que abarca ambas columnas
\maketitle

% --- RESUMEN (Abstract) ---
% En formato IEEE, el resumen va en negrita y al inicio.
\begin{abstract}
\textbf{Resumen}---Aquí presentas la síntesis de tu victoria. El objetivo de este proyecto es analizar señales de voz para identificar patrones asociados a la cordectomía. Se implementaron filtros digitales (FIR/IIR) y se analizaron las frecuencias fundamentales... 
\end{abstract}

% --- PALABRAS CLAVE ---
\begin{IEEEkeywords}
Procesamiento de Señales, Cordectomía, Filtros Digitales, Ingeniería Biomédica, MATLAB/Python.
\end{IEEEkeywords}

% --- EL CUERPO DEL TEXTO (Automáticamente en dos columnas) ---

\section{Introducción}
\IEEEPARstart{E}{l} análisis de la voz es una herramienta no invasiva fundamental... (Nota: `\IEEEPARstart` crea esa letra capital grande y elegante al inicio).

%Como se observa en el estudio de Betances et al. \cite{Betances2020}, el uso de sensores... (Aquí citas tu PDF de ejemplo).

\section{Metodología}
Para este análisis se utilizó una base de datos de...

\subsection{Adquisición de Datos}
Las señales fueron capturadas a una frecuencia de muestreo de...

\subsection{Pre-procesamiento}
Se aplicó un filtro pasa-banda de...

% Ejemplo de Ecuación (crucial para señales)
\begin{equation}
    H(z) = \sum_{k=0}^{M} b_k z^{-k}
    \label{eq:filtro}
\end{equation}

\section{Resultados}
La Figura \ref{fig:espectro} muestra el espectrograma comparativo...

\begin{figure}[h]
    \centering
    \includegraphics[width=\linewidth]{tu_imagen.png} % \linewidth se ajusta a la columna
    \caption{Comparación espectral. a) Voz sana. b) Voz con patología.}
    \label{fig:espectro}
\end{figure}

\section{Discusión}
Los resultados sugieren que...

\section{Conclusión}
El sistema propuesto demuestra...

% --- REFERENCIAS ---
\printbibliography

\end{document}
