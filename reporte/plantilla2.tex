% --- CLASE SOBERANA: IEEEtran ---
% 'journal' = Formato de revista (doble columna, denso).
% 'compsoc' = Optimizado para computación/ingeniería.
\documentclass[journal, compsoc, 12pt]{IEEEtran}

% --- ARSENAL DE IDIOMA Y FUENTE ---
\usepackage[utf8]{inputenc}
\usepackage[spanish, es-tabla]{babel} % 'es-tabla' usa "Tabla" en lugar de "Cuadro"
\usepackage{libertinus} % Tu fuente de autoridad
\usepackage[T1]{fontenc}
\usepackage{microtype}  % Micro-ajustes tipográficos esenciales para doble columna

% --- ARSENAL DE CITACIÓN(Vancouver/Numérico) ---
% IEEE usa citas numéricas [1], ideal para esto.
\usepackage[backend=biber, style=ieee, sorting=none]{biblatex}
\addbibresource{referencias.bib}

% --- UTILIDADES ---
\usepackage{graphicx}
\usepackage{float}
\usepackage{amsmath} % Para tus ecuaciones de señales
\usepackage{booktabs} % Para tablas profesionales
\usepackage[colorlinks=true, linkcolor=black, citecolor=black, urlcolor=blue]{hyperref}

% --- METADATOS DEL PAPER ---
\title{Análisis de Señales para la Detección de Patologías en Voz: Proyecto Cordectomía}

% En IEEE, los autores se ponen en un bloque especial
\author{Alfonso~Gamboa~Rubén, \\ Flores Monteros Edsel Yetlanezi% <- Este % evita espacios extra
\thanks{Manuscrito entregado el \today. Trabajo realizado para la materia de Análisis de Señales, Universidad Veracruzana.}}

% ==========================================================
%                  INICIO DEL PAPER
% ==========================================================
\begin{document}

% Crea el título que abarca ambas columnas
\maketitle

% --- RESUMEN (Abstract) ---
% En formato IEEE, el resumen va en negrita y al inicio.
\begin{abstract}
  Resumen en español
\end{abstract}

\begin{abstract}
  Abstract en inglés
\end{abstract}

% --- PALABRAS CLAVE ---
\begin{IEEEkeywords}
  Procesamiento Digital de Señales, Cordectom\'ia, Filtros Digitales, Formantes Vocales, Python.
\end{IEEEkeywords}

% --- EL CUERPO DEL TEXTO (Automáticamente en dos columnas) ---

\section{Introducción}
\IEEEPARstart{L}a cordectom\'ia es un proceso quir\'urgico que implica la extirpaci\'on parcial o total de las cuerdas vocales como se aprecia en la figura \ref{fig:cordectomia}; esto por cuestiones m\'edicas, como la extirpación de tumores. \cite{remacle2000endoscopic}


%Como se observa en el estudio de Betances et al. \cite{Betances2020}, el uso de sensores... (Aquí citas tu PDF de ejemplo).

\section{Objetivos}
  \subsection{Objetivo General}
  Lo que se busca es el desarrollar un sistema que nos permite poder analizar y filtrar una señal de una voz de pacientes con post-cordectomia. Recibimos nuestra señal, donde primeramente se filtrará y después la amplificaremos para poder escucharla con un buen volumen y que sea entendible

  \subsection{Objetivos Específicos}
  \begin{itemize}
  	\item Implementar t\'ecnicas de filtrado digital para reducir el ruido y analizar los componentes arm\'onicos de la señal de voz.
  	\item Comparar con ayuda de gráficas y la desiación estándar el cambio que hubo en la voz.
  	\item Validar el desepeño del software mediante diversas pruebas.
    \item Implementar con APIs de inteligencia artificial, para la recostrucción de una voz más parecida a la original del paciente.
  \end{itemize}



\section{Metodología}
La presente corresponde al diseño de un sistema enfocado en el desarrollo de una herramienta computacional para el análisis acústico de la voz en pacientes pre-cordectomía. La metodología se organiza en dos fases principales: (1) la definición de criterios de elegibilidad y análisis para la elección de las herramientas, y (2) el desarrollo del algoritmo de procesamiento y análisis de la señal.


\subsection{Adquisición de Datos}
Las señales fueron capturadas a una frecuencia de muestreo de...

\subsection{Pre-procesamiento}
Se aplicó un filtro pasa-banda de...

% Ejemplo de Ecuación (crucial para señales)
\begin{equation}
    H(z) = \sum_{k=0}^{M} b_k z^{-k}
    \label{eq:filtro}
\end{equation}

\section{Resultados}
La Figura \ref{fig:espectro} muestra el espectrograma comparativo...

\begin{figure}[h]
    \centering
    \includegraphics[width=\linewidth]{logo.png} % \linewidth se ajusta a la columna
    \caption{Comparación espectral. a) Voz sana. b) Voz con patología.}
    \label{fig:espectro}
\end{figure}

\section{Discusión}
Los resultados sugieren que...

\section{Conclusión}
El sistema propuesto demuestra...

% --- REFERENCIAS ---
\printbibliography

\end{document}
